% Documento LaTeX exemplo

% Cabecalho do documento
% Onde eh configurado todo o documento
%%%%%%%%%%%%%%%%%%%%%%%%%%%%%%%%%%%%%%%%%%%%%%%%%%%%%%%%%%
\documentclass{article}
\usepackage[brazil]{babel}


% Corpo do documento
% Onde eh escrito o texto
%%%%%%%%%%%%%%%%%%%%%%%%%%%%%%%%%%%%%%%%%%%%%%%%%%%%%%%%%%
\begin{document}

\title{Análise de variação de temperatura dos últimos cinco anos}
\author{Gustavo Gosling, Noá Giovanni}

\maketitle

\begin{abstract}
Texto Basico no LaTeX, tipo um resumo.

\end{abstract}

\section{Introdução}

Isso vai ser outra seção.
O começo de uma nova era. Não houve um dia que não chovia fogo.

E assim o Yago não viu o meu retorno. De todos os países que ele visitou,
 todos estavam com o tempo mudando.

\section{Metodologia}

Yago ajustou uma reta aos cinco últimos anos dos dados de temperatura
 média mensal para cada país que visitou.
Assim calculou-se a taxa de variação da temperatura recente.

A equação da reta é:

\begin{equation}
T(t)= a t + b,
\label{Yago}
\end{equation}

\noindent
onde $T$ é a temperatura (em graus Celsius), $t$ é o tempo (em anos),
 $a$ é o coeficiente angular e $b$ é o coeficiente linear (em graus Celsius)

Utilizamos a equação \ref{Yago} em um código Python para fazer um ajuste da reta com
o método dos mínimos quadrados

Outra equação doida é:

\begin{equation}
y = \int_\Omega x dx
\end{equation}

\end{document}